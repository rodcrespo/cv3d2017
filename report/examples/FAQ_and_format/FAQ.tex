\documentclass[pdf]{beamer}
\usetheme{hh}
 \input{Packages}
 \usepackage{animate}
\usepackage{hyperref}
\setbeamersize{text margin left=1.2em}

\usepackage{pgf,tikz}
\usetikzlibrary{arrows}
\definecolor{uuuuuu}{rgb}{0.26666666666666666,0.26666666666666666,0.26666666666666666}
\definecolor{qqqqff}{rgb}{0.,0.,1.}
\definecolor{ffqqqq}{rgb}{1.,0.,0.}

\def\reals{ { {\rm I \kern-0.15em R } } }
\DeclareMathOperator{\Tr}{Tr}
\newcommand{\argmax}[1]{\underset{#1}{\operatorname{arg}\!\operatorname{max}}\;}


\begin{document}



%% %Show toc at the begging of each section
%% \AtBeginSection[]
%% {
%% \begin{frame}{Outline}
%% \tableofcontents[currentsection] 
%% \end{frame}
%% }


%Title frame details
\title{\textcolor{black}{\strut  Recurring  deficits and questions in
    reports.}}
%% \author{\textcolor{white}{Anna Mikaelyan and Josef Bigun}\vspace{25mm}}
\author{
       Josef~Bigun,%~\IEEEmembership{Fellow,~IEEE}% <-this % stops a space
\\
  Halmstad University, \\Intelligent Systems Laboratory, \\ SE-30118 Sweden.\\
%% \IEEEcompsocitemizethanks{\IEEEcompsocthanksitem Authors are  with
%%   Halmstad University, Department of Embedded Intelligent Systems,  SE-30234 Sweden.\protect\\
% note need leading \protect in front of \\ to get a newline within \thanks as
% \\ is fragile and will error, could use \hfil\break instead.
  %% E-mail:
  josef.bigun@hh.se%
}
\date{ }
%\vspace{-25mm}\textcolor{white}\today}

%%Frame
\begin{frame}
\maketitle  %There is no need to issue title etc explicitly
             %...It is done in \maketitle
\end{frame}

\section {General}
\begin{frame}{General}
\begin{itemize}
\item {\bf  Can I write my article in another language than English?}\\
     No. To 
     express a scientific analysis in english is one of the outcomes
     of the course.
\item {\bf  What software should I use to produce the report?}\\
We recommend you to use pdflatex, which is available freely for the
most popular operative systems, as well as  directly as a web service
(e.g. search for sharelatex on internet). This is because writing  master
thesis report, presentation slides, and interacting with teachers of
our lab on these is  more
efficient if the production is done in pdflatex. You can also use
another software evidently as long as you know how to produce them.
\end{itemize} 
\end{frame}



\section{Introduction section} 
\begin{frame}{Recurring deficits } 
\begin{itemize}
  \item {\bf  Context and purpose not mentioned}\\
Please, tell that you were required to produce this report in the
course, and why so i.e. 
what it will supposedly demonstrate (your ability to find, read, and understand to a reasonable depth a scientific paper). 
\item {\bf  Another author's wording has been used for context and purpose...}\\
No, you must write it in your  way!  See  plagiarism below.
\item {\bf  Another author's wording has been used for citing a scientific reference...}\\
Again, No, you must write it in your  way! See  plagiarism below.
\end{itemize}
\end{frame} 

\section{Language and format} 
\begin{frame}{Language and format}
\begin{itemize}
\item {\bf Grammar is bad, spelling is bad...}   \\
We do not judge your language abilities in this course, although other courses may. HOWEVER, if the grammar and spelling is so seriously bad that your message 
and logics is not comperhensable, then you will have to raise its quality or else it will be rejected. We assume  that you will take every opportunity
in the rest of your studies to improve your ability in scientific writing.  A single report
is not sufficient.
\item {\bf I have listed references at the end but I have not commentted them in the text. Is this ok?} \\
No, you must comment/cite every single item appearing in the reference
list. Reference list must be demonstatively relevant, in particular  no item should be there as   decoration.
\end{itemize}
 \end{frame} 


\section{Plagiarism} 
\begin{frame}{Copy-paste,  Plagiarism}
\begin{itemize} 
\item {\bf Can I copy and paste text from Wikipedia?}\\
Absolutely NOT! Copying from ANYWHERE,  giving the impression that 
it is your words is plagiarism or piracy. This  can result in
exclusion from  the course and possibly  the school,  per bylaw of HH.
However,   citing is fine,  
provided that you  do it appropriately, e.g. quotations and with mention of the source.
\item {\bf I found a graphics material (picture, or drawing) in the
  internet.  Can I use it in my report? }
 \\
This is tricky and in        a gray zone. You have citation right, but
the material can be copyrighted.  In any case, if you use the material,  YOU MUST
DEMONSTRATIVELY write its source, in the caption. Else it is considered plagiarism again...with 
the same serious consequences as above. If you use too many such
graphics of the same source, you start to enter the black zone...
\end{itemize} 
\end {frame}

\section{Scientific references} 
\begin{frame}{Scientific references}
\begin{itemize}
\item {\bf How do you dare to refuse my citation to a reference  such as
  ``NEW SCIENTIST'', ``ILLUSTRERAD VETENSKAP'', ``WIKIPEDIA'',  ``Master-thesis report at OXFORD'' as non-scientfic ?} \\
No offense!. These are great sources to learn about science but they
are not peer reviewed scientific articles. They might even have a high
scientific quality. That said, you are allowed to make a reference to such an article. However, in this assignment we want you to recognize peer
reviewed scientific articles. Consequently, at least one item in the
reference list must be a peer reviewed scientific article. 
\end{itemize}
 \end{frame} 


\section{...Scientific articles} 
\begin{frame}{...Peer reviewed scientific articles}
\begin{itemize}
\item {\bf How do I recognize a peer reviewed scientific article then?  Is it written on it somewhere?}  \\
\begin{itemize}
\item  The fact is usually not written anywhere in the article.
\item Scientific articles cite (almost always) only other scientific
  articles. If the reference list of an article is dominated by
  web-links it is an indication for that the article is not
scientific. 
\item    Search engines specialized in scientific articles include
       \begin{itemize}
       \item     \url{http://scholar.google.com} \\
         \url{http://apps.webofknowledge.com}  HH library has a subsription
         service to this site offered to employees and students. You
         must be at HH OR 
          have connection permissions.
\item The sites \url{elsevier.com}, \url{springer.com}, \url{jstor.org} have
also (smaller) search facilities. These sites are also known
scientific article publishers. Patents will be considered as scientific.
\end{itemize}
\item Finally, recognizing peer-reviewed scientific papers is one of the
  outcomes of this course.
\end{itemize}
\end{itemize}
\end{frame}



%jbj




\end{document}
